\documentclass{article}
\usepackage[utf8]{inputenc}
\usepackage{listings}

\title{Livrable 1}
\author{BAV4}
\date{17/02/2025}

\begin{document}

\section{choix du niveau d’infrastructure}

\section{architecture envisagée et plan d’adressage}

\section{liste des logiciels choisis}

\section{critères de choix du Wiki}
    \subsection{Facilité d’utilisation}
    mdBook utilise le langage Markdown, qui permet d’écrire du texte de manière simple et intuitive. Avec Markdown, on peut facilement mettre en forme des titres, des listes, du code et des liens sans se soucier de balises compliquées. Cela rend la rédaction beaucoup plus rapide et accessible pour toute l’équipe.\\


    De plus, comme il s’agit de fichiers texte, on peut les éditer avec n’importe quel éditeur, que ce soit en ligne de commande ou avec des applications graphiques. Cela offre une grande flexibilité pour travailler sur différents environnements.

    \subsection{Collaboration simplifiée grâce à Git}
    Un des atouts majeurs de mdBook est son intégration avec Git. En utilisant un dépôt Git, tous les membres de l’équipe peuvent collaborer facilement sur le même projet. Ils peuvent cloner le dépôt, créer des branches pour leurs modifications, et soumettre des mises à jour via des pull requests.\\

Cela permet de suivre l’historique des changements et de voir qui a modifié quoi, quand et pourquoi. En cas d’erreur, on peut revenir facilement à une version précédente. Cela garantit une transparence totale et évite les conflits d’édition.

\subsection{Déploiement en local et sécurité des données}
Contrairement à d’autres solutions qui nécessitent un serveur web complexe ou une base de données, mdBook génère des fichiers HTML statiques. Ces fichiers peuvent être ouverts localement dans un navigateur ou hébergés sur un serveur léger sans avoir besoin de logiciels supplémentaires.\\

Cette approche permet de garder le wiki en local, garantissant ainsi la confidentialité des données sensibles du projet réseau. Cela réduit aussi les risques liés à la sécurité, car aucune information n’est exposée sur Internet.

    \subsection{Interface claire et moderne}

    mdBook offre une interface claire, inspirée des documentations techniques modernes. La navigation est simple, avec un menu latéral qui liste les chapitres et sous-chapitres. Cela permet aux utilisateurs de se repérer facilement et de trouver rapidement les informations dont ils ont besoin.\\

De plus, mdBook intègre un moteur de recherche qui permet de rechercher des mots-clés dans l’ensemble du wiki, ce qui est très pratique pour naviguer dans une documentation volumineuse.

    \subsection{Maintenance facile et légèreté}
    Un autre avantage de mdBook est sa légèreté. Contrairement à d’autres solutions de wiki qui nécessitent des bases de données ou des environnements serveurs complexes, mdBook ne nécessite que Rust pour fonctionner.\\

Cela simplifie énormément la maintenance : pas besoin de gérer des mises à jour de logiciels serveurs ou de bases de données. Tout est basé sur des fichiers texte, ce qui facilite les sauvegardes et les restaurations en cas de besoin.

\subsection{Flexibilité et personnalisation}

mdBook est hautement personnalisable. On peut modifier le thème par défaut pour l’adapter aux besoins du projet ou à l’identité visuelle de l’équipe. Il est aussi possible d’ajouter des extensions et des plugins pour inclure des diagrammes, du code interactif ou d’autres fonctionnalités avancées.\\

Cette flexibilité permet d’adapter le wiki aux besoins spécifiques de mon projet réseau tout en restant simple à utiliser et à maintenir.

\section{solutions mises en œuvre pour la communication sécurisée}

\section{exemple de Latex}

\subsection{Exemples de liste}
\begin{enumerate}
    \item Un item de liste
    \item Un autre item de liste
    \item Un autre item de liste qui contient une liste 
        \begin{itemize}
        \item  Un item de liste a puce
        \item  Un autre item de liste a puce
        \item  encore un autre item de liste a puce
        \end{itemize}
\end{enumerate}

\subsection{Pour le texte}
Pour sauter a la ligne 2 possibiités : 

On laisse une ligne entre les deux phrases; ou on utilise ca : \\
Qui permet de passer a la suivante

listings : librairie pour ecrire du code
\begin{lstlisting}
    ldd /usr/bin/php8.4
\end{lstlisting} 

\end{document}