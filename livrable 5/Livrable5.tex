\documentclass{article}

\usepackage[utf8]{inputenc}
\usepackage{amsmath}
\usepackage{graphicx}

\title{Livrable 5}
\author{
Schicke Samuel, Burellier Loucas, Amberny Peran, \\
Krainik-Saul Vladimir, Barnouin Clement
}
\date{\today}

\begin{document}

\include{bav4 style file}

\maketitle


\section{Introduction}
MiniCoffee est un groupe français spécialiste de l’univers du café, connu notamment pour ses machines à café en libre service. Pour l’année 2025, l’entreprise souhaite mettre à jour son infrastructure réseau interne en ajoutant : 
\begin{itemize}
    \item Divers serveurs d’utilité interne pour les employés et l'équipe informatique;
    \item Un réseau invité pour permettre à ses fournisseurs d’utiliser du matériel informatique sur place;
    \item Divers serveurs accessibles en ligne (site Web, serveur DNS public);
    \item Une meilleure communication entre ses machines à café et son infrastructure, qui a été un des points faibles de l’entreprise ces dernières années.
\end{itemize}
Pour cette tâche, MiniCoffee a fait appel à BAV4, notre équipe d’étudiants de l’IUT2 Informatique de Grenoble.

\section{Architecture}
L'architecture de notre réseau n'a pas énormément changé. 
Les seules modifications apportées au réseau sont : 
\begin{itemize}
    \item Passage d'un LAN à un VLAN pour une meilleure segmentation du réseau.
    \item Les adresses IP internes se terminent par 1XX.
    \item Les adresses IP externes se terminent par XX.
\end{itemize}

\begin{figure}
    \centering
    \includegraphics[width=1.2\textwidth, trim=0 0 0 2.3cm, clip]{../assets/Architecture.drawio.png}
    \caption{Architecture réseau de MiniCoffee}
\end{figure}

\clearpage

\section{Ressources Matérielles utilisées}
This is the conclusion section. Summarize your findings or the main points of your document here.

\end{document}